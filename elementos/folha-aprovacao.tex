\tccOuNao{ % Folha de aprovação, para o caso de TCC
    \begin{Center}
        \autorTrabalho

        \vspace{\baselineskip}
        \textbf{\tituloTrabalho\separadorTrabalho}\subtituloTrabalho
        
        \vspace{\baselineskip}
        Este Trabalho de Conclusão de Curso foi julgado adequado para obtenção do Título de \tituloCurso\space e aprovado em sua forma final pelo Curso \nomeCurso.

        \vspace{\baselineskip}
        \cidadeDefesa, \diaDefesa\space de \mesDefesa\space de \anoDefesa.

        \vspace{0.75cm}
        \rule{6.5cm}{1pt}\\
        \coordenadorTitulacao\space \coordenador\\
        Coordenador(a) do Curso

        \vspace{\baselineskip}
        \textbf{Banca examinadora:}

        \vspace{0.75cm}
        \rule{6.5cm}{1pt}\\
        \orientadorTitulacao\space \orientador\\
        Orientador(a)\\    
        \orientadorInstituicao

        % Verifico se o coorientador está preenchido, para imprimir
        \isEmpty{\coorientador}{
            \vspace{0.75cm}
            \rule{6.5cm}{1pt}\\
            \coorientadorTitulacao\space \coorientador\\
            Coorientador(a)\\    
            \coorientadorInstituicao
        }

        \vspace{0.75cm}
        \rule{6.5cm}{1pt}\\
        \bancaUmTitulacao\space \bancaUm\\
        Avaliador(a)\\    
        \bancaUmInstituicao

        \vspace{0.75cm}
        \rule{6.5cm}{1pt}\\
        \bancaDoisTitulacao\space \bancaDois\\
        Avaliador(a)\\    
        \bancaDoisInstituicao
    \end{Center}
}{ % Folha de certificação, para o caso de teses ou dissertações
    \begin{Center}
        \autorTrabalho

        \textbf{\tituloTrabalho\separadorTrabalho}\subtituloTrabalho
    \end{Center}
    
    \vspace{\baselineskip}
    O presente Trabalho de \modalidadeCurso\space foi avaliado e aprovado por banca examinadora composta pelos seguintes membros:

    \begin{Center}
        \vspace{\baselineskip}
        \rule{6.5cm}{1pt}\\
        \orientadorTitulacao\space \orientador\space - orientador(a)\\    
        \orientadorInstituicao

        % Verifico se o coorientador está preenchido, para imprimir
        \isEmpty{\coorientador}{
            \vspace{\baselineskip}
            \rule{6.5cm}{1pt}\\
            \coorientadorTitulacao\space \coorientador\space - coorientador(a)\\    
            \coorientadorInstituicao
        }

        \vspace{\baselineskip}
        \rule{6.5cm}{1pt}\\
        \bancaUmTitulacao\space \bancaUm\space - examinador(a)\\    
        \bancaUmInstituicao

        \vspace{\baselineskip}
        \rule{6.5cm}{1pt}\\
        \bancaDoisTitulacao\space \bancaDois\space - examinador(a)\\    
        \bancaDoisInstituicao

        \vspace{\baselineskip}
        \rule{6.5cm}{1pt}\\
        \bancaTresTitulacao\space \bancaTres\space - examinador(a)\\    
        \bancaTresInstituicao
    \end{Center}

    \vspace{\baselineskip}

    Certificamos que esta é a \textbf{versão original e final} do trabalho de conclusão que foi julgado adequado para obtenção do título de \tituloCurso\space em \nomeCurso.
    
    \begin{Center}
        \vspace{1cm}
        \rule{6.5cm}{1pt}\\
        \coordenadorTitulacao\space \coordenador\\
        Coordenador(a) do Programa de Pós-Graduação

        \vspace{1cm}
        \rule{6.5cm}{1pt}\\
        \orientadorTitulacao\space \orientador\\
        Orientador(a)

        \vspace{\fill}
        \cidadeDefesa, \anoEntrega.
    \end{Center}
}