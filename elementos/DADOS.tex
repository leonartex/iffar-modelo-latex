%IMPORTANTE
\newcommand{\isTCC}{1} % Se o seu trabalho for TCC, coloque qualquer valor dentro, um letrinha só, se quiser. Mas se o seu trabalho não for um TCC, ou seja, é tese ou dissertação, deixe em branco. Esse comando é usado para verificar e então configurar algumas páginas, como capa, folha de rosto e folha de aprovação/certificação
\newcommand{\tccOuNao}[2]{%
    \ifnum \value{\isTCC}>0
        #1
    \else
        #2
    \fi
}

% Dados de identificação do trabalho
\newcommand{\tituloTrabalho}{Modelo LaTeX}
\newcommand{\separadorTrabalho}{:\space} % Aqui você identifica o caractere que irá separar o título do subtítulo. Eu já vi trabalho que não utiliza dois pontos, então sei lá, deixei para permitir configuração, recomendo conversar com quem te orienta. Se não possuir subtítulo, deixar vazio, em branco (sem espaço)
\newcommand{\subtituloTrabalho}{como é gostoso usar um template redondinho} % Se não possuir subtítulo, deixar vazio, em branco (sem espaço)

% O seu nome completo
\newcommand{\autor}{Insira o Seu Nome Completo}

% Os dados sobre a apresentação da defesa
\newcommand{\cidadeDefesa}{Insira a cidade onde ocorre a defesa}
\newcommand{\anoEntrega}{Ano de entrega} % O ano em que o trabalho foi entregue
\newcommand{\diaDefesa}{Dia}
\newcommand{\mesDefesa}{Mês por extenso}
\newcommand{\anoDefesa}{Ano de defesa} % O ano em que o trabalho foi defendido (atente-se que pode ter diferença)
\newcommand{\tituloCurso}{Insira o título de formação do seu curso}
\newcommand{\nomeCurso}{Insira o nome do seu curso}

% Os dados de quem te orientou
\newcommand{\orientador}{Insira o nome do(a) orientador(a)}
\newcommand{\orientadorTitulacao}{Prof.(a) Dr.(a)}
\newcommand{\coorientador}{Insira o nome do(a) coorientador(a)} % Deixe em branco se não possuir
\newcommand{\coorientadorTitulacao}{Prof.(a) Dr.(a)}

% Os dados sobre a sua banca
\newcommand{\bancaUm}{Insira o nome da primeira pessoa da banca}
\newcommand{\bancaUmTitulacao}{Prof.(a) Dr.(a)}
\newcommand{\bancaDois}{Insira o nome da segunda pessoa da banca}
\newcommand{\bancaDoisTitulacao}{Prof.(a) Dr.(a)}
\newcommand{\bancaTres}{Insira o nome da terceira pessoa da banca} % Deixe em branco a terceira pessoa da banca se for um TCC (ou tanto faz, não vai ser impresso de qualquer forma) 
\newcommand{\bancaTresTitulacao}{Prof.(a) Dr.(a)}