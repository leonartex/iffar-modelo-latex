\chapter*{Resumo}\myIdxResumoVernac

% O \noindent é necessário por causa que, como eu configurei para indentar o primeiro parágrafo no texto, como exigido, o LaTeX iria normalmente indentar o parágrafo
% Só escrever o teu resuminho, sem estresse
\noindent [Inserir o resumo que representa uma síntese do trabalho e cujo texto é escrito na língua vernácula, de modo objetivo, sucinto e claro. Deve ser digitado como um bloco único de parágrafo e apresentar uma sequência de frases concisas, esclarecendo: a delimitação da temática/assunto abordado; o objetivo principal da pesquisa; a questão norteadora ou hipótese(s) do trabalho, bem como o viés teórico-metodológico que o fundamentou. Deve constar também o método e os procedimentos técnicos utilizados na obtenção e na análise dos dados; apontando principais resultados e pontos de discussão. O resumo não deve ser uma enumeração de tópicos, nem conter citações. Recomenda-se isar o verbo na voz ativa e na terceira pessoa do singular. O uso do verbo na primeira pessoa depende do diálogo com o(a) orientador(a) e o Programa de Pós-graudação no qual o Trabalho Acadêmico-Científico está sendo desenvolvido. \textbf{Seu texto deve ser digitado com espaçamento entre linhas simples}\footnote{O próprio Guia de Normalização contradiz isso. E a NBR 14724 não cita o resumo com um dos caso de espaçamento simples.} e fonte de tamanho 12. Evitar o uso de fórmulas, diagramas e siglas sem a respectiva designação. O resumo deve conter de 150 a 500 palavras e ser elaborado em conformidade com a NBR 6028:2021. Abaixo do resumo, deve-se deixar uma linha em branco e colocar a expressão ``Palavras-chave'', seguida de dois pontos e um espaço. Em seguida, deve-se informar as palavras-chave, especificando de três a cinco delas (constituída por termos extradídos de Thesaurus da área ou expressões significativas ao texto, preferencialmente que não sejam coincidentes com as palavras usadas no título do trabalho para favorecer sua busca por mecanismos de recuperação). Exceto quando ela se constituir como um substantivo próprio ou nome científico, cada palavra-chave é iniciada por letra minúscula, seguida de ponto e vírgula, e de um espaço, sendo a última finalizada por um ponto]. 

\vspace{\baselineskip} % Fazer o espacinho entre as duas partes

\noindent \textbf{Palavras-chave:} palavra-chave 1; palavra-chave 2; palavra-chave 3.

% Recomendação do Guia de Normalização de iniciar cada elemento pré-textual e textual no anverso da página
\OnesideTwoside{\clearpage}{\cleardoublepage}