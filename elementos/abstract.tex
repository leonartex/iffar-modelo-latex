% É exatamente o mesmo código do resumo
% Só fico na dúvida, será que o abstract tem que ser escrito em itálico, já que é todo em outro idioma? Se sim, é só englobar todas as partes com texto (sem o \noindent) em um \textit{}
\chapter*{Abstract}\myIdxAbstract

% O \noindent é necessário por causa que, como eu configurei para indentar o primeiro parágrafo no texto, como exigido, o LaTeX iria normalmente indentar o parágrafo
% Só escrever o teu abstractzinho, sem estresse
\noindent [Inserir o abstract, que é o resumo informativo redigido em parágrafo único, o qual deve ser apresentado traduzido para outros idiomas, como inglês, exemplificado nesse caso. Alguns programas de pós-graduação podem exigir outras línguas além do inglês. O abstract deve possuir a mesma estrutura e formato do resumo feito na língua vernácula. As palavras-chave também são traduzidas para a respectiva língua estrangeira adotada e devem ser dispostas seguindo-se as mesmas orientações anteriores. Ou seja, elas devem ser colocadas abaixo do texto, após a inserção de uma linha em branco, sendo precedias pela expressão ``Keywords'', seguida de dois pontos, de um espaço e, apó, deve ser inseria cada palavra-chave iniciada por letra minúscula, separadas entre si por ponto e vírgula, sendo a última finalizada por um ponto. Deve ser elaborado em conformidade com a NBR 6028:2021].

\vspace{\baselineskip} % Fazer o espacinho entre as duas partes

\noindent \textbf{Keywords:} keyword 1; keyword 2; keyword 3.

% Recomendação do Guia de Normalização de iniciar cada elemento pré-textual e textual no anverso da página
\OnesideTwoside{\clearpage}{\cleardoublepage}