% Crio um elemento utilizado para listar cada símbolo e seu significado
% Obs.: É a mesma lógica da lista de siglas. Ajuste o tamanho do comprimento da "caixinha" do símbolo conforme a sua necessidade
% Fonte: https://tex.stackexchange.com/a/508154
\newcommand\simboloTamanho{5ex} % Comprimento do espaço onde fica o símbolo
\newcommand\simboloGap{1ex} % Comprimento do vão entre símbolo e seu nome, para dar uma folguinha no tamanho máximo
\newcommand\simboloNomeTamanho{\dimexpr\linewidth-\simboloTamanho-\simboloGap\relax}
\newcommand\simbolo[2]{\noindent\parbox[t]{\simboloTamanho}{#1\strut}%
  \hspace{\simboloGap}%
  \parbox[t]{\simboloNomeTamanho}{#2\strut}}

\chapter*{Lista de símbolos}
% Para adicionar uma abreviatura ou sigla na lista apenas precisa adicionar a sigla dentro do \makebox igual às 

\simbolo{\S}{Parágrafo}

\simbolo{$\Sigma$}{Somatória}

\simbolo{$\sum_{n = 1}^{\infty}$}{Matemágica}

\simbolo{xxxxx}{Lorem ipsum dolor sit amet, consectetur adipiscing elit. Curabitur et neque metus. Ut volutpat sem quis lacus rutrum, a viverra odio tincidunt}