% Crio um elemento utilizado para listar cada abreviatura e sigla
% Obs.: Dependendo do tamanho das siglas utilizadas, você pode querer aumentar ou diminuir o espaço entre a sigla e o seu significado, porém, mantendo o alinhamento, para isto, você modifica o comando \newcommand\siglaTamanho{10ex}, substituindo o 10ex pelo tamanho que você deseja
% O tamanho 10ex suporta uma sigla de até mais ou menos 10 letras, visto que a unidade utiliza como referência o tamanho da letra "x". Você também pode utilizar a unidade "em", que utiliza como referência a letra "m", se quiser.
% Fonte: https://tex.stackexchange.com/a/508154
\newcommand\siglaTamanho{10ex} % Comprimento do espaço onde fica a sigla
\newcommand\siglaGap{1ex} % Comprimento do vão entre sigla e seu significado, para dar uma folguinha no tamanho máximo
\newcommand\significadoSiglaTamanho{\dimexpr\linewidth-\siglaTamanho-\siglaGap\relax}
\newcommand\sigla[2]{\noindent\parbox[t]{\siglaTamanho}{#1\strut}%
  \hspace{\siglaGap}%
  \parbox[t]{\significadoSiglaTamanho}{#2\strut}}

\chapter*{Lista de abreviaturas e siglas}
% Para adicionar uma abreviatura ou sigla na lista apenas precisa adicionar a sigla dentro do \makebox igual às 

\sigla{IBGE}{Instituto Brasileiro de Geografia e Estatística}

\sigla{T}{Teste}

\sigla{xxxxxxxxxx}{Lorem ipsum dolor sit amet, consectetur adipiscing elit. Curabitur et neque metus. Ut volutpat sem quis lacus rutrum, a viverra odio tincidunt}