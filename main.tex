%%%%%%%%%%%%%%
% Modelo XXXXXXXXXXXXXXX
% Autor: Leonardo Garcez Dalenogari Alba
% Ano: 2022
% Contato: leonardo.2017007421@aluno.iffar.edu.br

% IMPORTANTE: 
%   O modelo utiliza o compilador XeLaTex (talvez possam ser utilizados outros, mas não é garantido o correto funcionamento, visto que seria necessário testar, porém, eu sei que alguns vão apresentar problemas na hora de compilar, como o pdflatex)
%   Se você estiver utilizando o Overleaf, com o documento aberto, clique em Menu, no canto superior esquerdo, e selecione o XeLaTex como compilador
%   Também é utilizado o biber para gerenciar as referências bibliográficas
%%%%%%%%%%%%%%

\documentclass[
    12pt, % Tamanho de fonte definido para 12, se você precisar utilizar um texto em tamanho 10, utilize o comando \footnotesize, que será impresso em tamanho 10
    oneside,
    %twoside, % A opção permite gerar o documento utilizando o verso das páginas, porém, ela não indica à impressora que você está utilizando isso ou que ela deva imprimir nos dois lados das folhas (fonte: https://en.wikibooks.org/wiki/LaTeX/Document_Structure)
    %openany,
    %openright,
    a4paper,
    english,
    brazil
]{report}

% Pacotes que envolvem codificação do conteúdo. Essenciais.
\usepackage[T1]{fontenc}
% \usepackage[utf8]{inputenc} % Não é necessário se você estiver utilizando um compilador LaTeX baseado já em UTF-8

% Responsável pela quebra silábica de palavras no final de linha de forma correta para o português
\usepackage[brazil]{babel}
% O pacote babel recomenda o uso do csquotes, para garantir que textos citados estejam escritos corretamente de acordo com as regras do seu idioma.
\usepackage[portuguese=brazilian]{csquotes}

%% Pacotes para formatação de fonte do documento
\usepackage{fontspec}
% Pode ser Times New Roman:
% \setmainfont{Times New Roman} 
% Ou pode ser Arial, também:
\setmainfont{Arial}

% Pacote responsável por adicionar símbolos matemáticos. Símbolos, por exemplo, também necessários para o correto referenciamento de leis e decretos
\usepackage{amssymb}
% Só que também é importante realizar a configuração para adicionar os comandos que imprimem esses símbolos associados aos caracteres que imprimem, assim, você pode adicionar o símbolo diretamente no texto
\usepackage{newunicodechar} % Pacote que permite definir novos caracteres
\usepackage{config/unicode-char} % Arquivo de configuração com a listagem desses caracteres

%% Pacotes para formatação de conteúdo na página
\usepackage[left=3cm,top=3cm,right=2cm,bottom=2cm]{geometry} % Pacote responsável por formatar as dimensões das margens das páginas.
\usepackage{setspace} % Adiciona opções de espaçamento de texto
\usepackage{ragged2e} % Adiciona opções mais completas para alinhamento de texto
\usepackage{indentfirst} % Faz com que o primeiro parágrafo de cada seção possua identação também

% Pacote que permite criar estilos para o cabeçalho
\usepackage{fancyhdr}
\usepackage{config/cabecalho} % Cabeçalho customizado para o modelo

%% Adiciono os pacotes para configurar e estilizar o sumário e as seções
% \usepackage{tocloft}
\usepackage{titletoc} % Não consegui criar pelo titletoc a parte de todas as seções estarem alinhadas, então vou usar o tocloft
\usepackage{titlesec}
\usepackage{alphalph} % Permite lidar com contadores que possuam valor maior que 26 utilizando o alfabeto (o número de letras), ou seja, se passar de 26, vai utilizar AA, AB, AC... Nativamente o LaTeX não realiza esse controle
\usepackage{config/apendice-anexo} % Criação dos comandos para apêndice e anexo

%% Agora customizo as seções e sumário para seguirem as normas do IFFar (itens adicionais do sumário, como apêndice e anexos, também são configurados nesses arquivos)
\usepackage{config/sumario} % Importa o sumário feito para o modelo
%% Adiciona a possibilidade de estilizar os títulos de seções, subseções e afins
\usepackage{config/titulo-sections} % Importa as estilizações feitas das seções e subseções para o modelo

% Pacote para facilitar a customização de listas numeradas, que serão utilizadas para as alíneas e subalíneas
\usepackage{enumitem}
\usepackage{config/enumerate}

\usepackage{float} % Adiciona configurações adicionais para elementos float (figuras, tabelas, quadros, etc.)
\usepackage{newfloat} % Permite criar floats, como o caso dos quadros, de forma prática
\DeclareFloatingEnvironment{quadro} % Crio o elemento float Quadro
\usepackage{tabularx} % Tabelas no LaTeX são limitadas em algumas características, este pacote resolve algumas dessas falhas
\usepackage{graphicx} % Adiciona a capacidde de adicionar figuras, além de outras funcionalidade que podem ser utilizadas em quadros e tabelas.
% Figuras podem ser imagens (SVG, por padrão, não funciona), mas documentos PDF também são compatíveis
\graphicspath{ {./img/} } % Adiciona um caminho padrão para arquivo
% Permite modificar a apresentação das legendas (caption) de figuras, quadros e tabelas
\usepackage[labelsep=endash, font=singlespacing, justification=centering]{caption} % Modifica a separação da numeração da legenda com a descrição dela
% Removo a contabilização de capítulo na numeração das figuras, tabelas e quadros (isso ocorre ao utilizar o documentclass de book ou report)
\counterwithout{figure}{chapter} 
\counterwithout{table}{chapter}
\counterwithout{quadro}{chapter}

%% Pacotes para adicionar o uso de citações e referências no padrão ABNT
% Permite customizar as notas de rodapé. Necessário para a identação das notas
\usepackage[hang]{footmisc}
\usepackage{config/rodape} % Configurações das notas de rodapé
% Usa o biber por ser uma versão mais parruda de biblatex, que é uma versão mais parruda do bibtex, ou algo assim. É importante por resolver uma quantia de erros por conta de caracteres, principalmente na URL das referências
\usepackage[backend=biber, style=abnt]{biblatex}
\addbibresource{bibliografia.bib} % Você pode adicionar mais de um arquivo.bib, apenas adicione outra linha com o nome do segundo arquivo
\usepackage{config/bibliografia-abnt} % Ajustes necessários por causa que o pacote biblatex-abnt não foi atualizado para algumas das mais novas normas da ABNT

%Pacote responsável por permitir colocar url. Ele cria link de atalho no sumário, citações e referência de figuras, tabelas, etc.
\usepackage[hidelinks]{hyperref}
\hypersetup{
    unicode=true
}
\usepackage{url}

\usepackage{lipsum} % Pacote meramente cosmético, para testes. Remova ele quando for utilizar o modelo.

\begin{document}
\onehalfspacing
\justify

\setlength\parindent{1.25cm} % Configura o espaçamento da identação dos parágrafos
\setlength{\parskip}{1.241pt} % Configura o espaço separação entre parágrafos.

\begin{SecoesNaoNumeradas}
\tableofcontents
\end{SecoesNaoNumeradas}
% \clearpage

% Seção ilustrativa sobre seções e subseções
% \chapter{Teste}
\begingroup
  \chapter{Exemplo de seção primária} 
  \lipsum[1-1]

  \section{Exemplo de seção secundária}
  \lipsum[2-4]

  \subsection{Exemplo de seção terciária}
  \lipsum[5-5]

  \subsubsection{Exemplo de seção quaternária}
  \lipsum[6-6]

  \paragraph{Exemplo de seção quinária} % Utiliza-se paragraph para o nível hierarquico inferior à subsubsection. Contudo, foi necessário customizá-lo para apresentar corretamente como um nível de seção
  \lipsum[7-7]

  \paragraph{Exemplo de título com indicação numérica que, ao ocupar mais de uma linha, deve ser, a partir da segunda linha, alinhado abaixo da primeira letra da primeira palavra do título}
  \lipsum[8-8]
\endgroup

\begingroup
  \chapter{Exemplos de elementos comuns}
\section{Notas de rodapé}
  \lipsum[10]\footnote{\lipsum*[11]}

\section{Alíneas e subalíneas}
  \lipsum[12]
  \begin{alinea}
    \item a matéria da alínea começa por letra minúscula, exceto quando se tratar de     substantivos próprios, e termina em ponto e vírgula, com exceção da última, que termina em ponto final; 
    \item o trecho final da seção correspondente, anterior às alíneas, termina em
    dois pontos;
    \item asdasdsada: 
    \begin{subalinea}
      \item a matéria da alínea começa por letra minúscula, exceto quando se tratar de     substantivos próprios, e termina em ponto e vírgula, com exceção da última, que termina em ponto final; 
      \item o trecho final da seção correspondente, anterior às alíneas, termina em
      dois pontos;
    \end{subalinea}
    \item sdsdsdaa
  \end{alinea}
  
  \lipsum[13]

  % \subsection{Outros exemplos de listas numeradas}
    % \begin{leiParagrafo}
    %   \item a matéria da alínea começa por letra minúscula, exceto quando se tratar de     substantivos próprios, e termina em ponto e vírgula, com exceção da última, que termina em ponto final; 
    %   \item o trecho final da seção correspondente, anterior às alíneas, termina em
    %   dois pontos;
    %   \item asdasdsada
    %   \item sdsdsdaa
    % \end{leiParagrafo}

\section{Equações e fórmulas}

\section{Figuras, tabelas e quadros}
  --- Falar que figure, table e quadro são apenas invólucros, que qualquer coisa pode ser adicionada dentro, que apenas indica que o elemento possuirá seu respectivo tratamento diferenciado na hora de listar e apresentar como tal
  
\subsection{Figuras}
  --- Falar os formatos de imagem suportados
  \begin{figure}[H]
    \Centering\singlespacing
    \caption{As dimensões da comunicação organizacional}
    \label{figura:marca-iffar}
    % Podem ser utilizados vários argumentos para adicionar a figura em seu devido tamanho, como: scale; width; ou height. No caso de scale, a escala, adiciona o valor em decimal representando a porcentagem da escala, já width ou height utiliza-se uma unidade de medida, como cm, px ou pt.
    \includegraphics[scale=0.35]{marca-iffar.png}\\
    \footnotesize
    Fonte: \textcite{iffar-identidade-visual-2021}.
  \end{figure}
  
\subsection{Tabelas}
\begin{table}[H]
  \Centering\singlespacing

  \caption{Posições do IFFar no \textit{ranking} do painel de dados da LAI}
  \label{tabela:cgu-iffar}
  \begin{tabularx}{12cm}
    {X r r} % Definição das colunas
    \hline
    \multicolumn{1}{c}{\textbf{Categoria}}    &   
    \multicolumn{1}{c}{\textbf{Posição}}  & 
    \multicolumn{1}{c}{\textbf{Informação adicional}} \\
    \hline

    Transparência ativa &
    252\textordmasculine{} &
    24,49\% em cumprimento \\
    %\hline

    Pedidos recebidos &
    192\textordmasculine{} &
    807 pedidos \\
    %\hline

    Tempo médio de resposta para pedidos de informação       &   
    154\textordmasculine{} &
    14,91 dias \\
    \hline

  \end{tabularx}

\hspace{\fill}

\footnotesize

Fonte: Gaba guba.
\end{table}

\subsection{Quadros}
--- Falar que é uma tabela com bordinha nos lados
\begin{quadro}[H]
  \Centering\singlespacing

  \caption{Posições do IFFar no \textit{ranking} do painel de dados da LAI}
  \label{quadro:cgu-iffar}
  \begin{tabularx}{12cm}
    {|X|r|r|} % Definição das colunas
    \hline
    \multicolumn{1}{|c|}{\textbf{Categoria}}    &   
    \multicolumn{1}{c}{\textbf{Posição}}  & 
    \multicolumn{1}{|c|}{\textbf{Informação adicional}} \\
    \hline

    Transparência ativa &
    252\textordmasculine{} &
    24,49\% em cumprimento \\
    %\hline

    Pedidos recebidos &
    192\textordmasculine{} &
    807 pedidos \\
    %\hline

    Tempo médio de resposta para pedidos de informação       &   
    154\textordmasculine{} &
    14,91 dias \\
    \hline

  \end{tabularx}

\hspace{\fill}

\footnotesize

Fonte: Gaba guba.
\end{quadro}
    

  Conforme o quadro no Apêndice \ref{apendice:teste} e o Anexo \ref{anexo:teste}. Citando \textcite{brasil-lei-11.892-2008}
  
  Testesddsdfdfsd

\endgroup

%% Aqui são adicionados as referências, apêndices e anexos. É essencial que eles estejam dentro desse ambiente, para formatar corretamente o título da seção deles
\begin{SecoesNaoNumeradas}
%% Aqui é impressa a lista de referências bibliográficas
\begingroup
  \begin{FlushLeft} % As referências devem apresentar justificação à esquerda
    % heading=bibintoc imprime o nome da seção das referências sem numeração
    % title=Referências modifica o nome da seção, que normalmente seria "Bibliografia", por conta que é o padrão utilizado pelos documentclass book e report
    \printbibliography[heading=bibintoc, title=Referências]
  \end{FlushLeft}
\endgroup

%%
% Se quiser mudar o comportamento do LaTeX de criar uma nova página para cada capítulo, no caso de Apêndices e Anexos, envelope o capítulo em um group e utilize o comando: \let\clearpage\relax, como demonstrado abaixo
\begingroup
\let\clearpage\relax
\apendice{Um exemplo de apêndice}\label{apendice:teste}
\lipsum[6-7]
\endgroup

\begingroup 
\let\clearpage\relax
\anexo{Um exemplo de anexo}\label{anexo:teste}
\lipsum[6-7]
\endgroup
\end{SecoesNaoNumeradas}

\end{document}