\chapter{Citações e referências}\label{capitulo:referencias}
Uma das características mais positivas do \LaTeX{} é a capacidade de gerenciar o uso de referências e citações, que podem ser referenciações de elementos presentes em seu texto ou citações de referências bibliográficas. E, para referências bibliográficas\footnotemark, este modelo utiliza o pacote \verb|biblatex|, junto com a sua opção de formatação da ABNT. Então, este capítulo busca trazer um conjunto de exemplos do uso dessas referências, também levando em consideração o \verb|biblatex|, para a correta formatação. O pacote não tem como possuir a formatação de todos os inúmeros tipos de documentos padronizados pela ABNT, sendo necessário, às vezes, adaptar para conseguir a formatação desejada.
  \footnotetext{Neste capítulo são apresentados exemplos de referências considerando especificamente o uso no \LaTeX{}, contudo, se tiver maiores dúvidas sobre referências, é essencial que você consulte o Guia de Normalização de trabalhos do IFFar e a norma ABNT NBR 6023:2018. Os dois documentos trazem um conjunto de exemplos de referências, mas, principalmente, o Guia também traz uma explicação e um conjunto de exemplos de como as citações devem ser realizadas.}

  Na pasta \texttt{docs} deste modelo, também estão presentes um conjunto de documentos que podem ser usados para auxiliar no uso do \texttt{biblatex}: o primeiro é uma folha de consulta rápida trazendo os principais comandos utilizados no \texttt{biblatex}, o \texttt{biblatex-cheatsheet.pdf} \cite{pdf:biblatex-cheatsheet}; o segundo é a documentação do \\\texttt{biblatex-abnt}, com um conjunto de exemplos \cite{pdf:biblatex-abnt}; o terceiro é a documentação do próprio \texttt{biblatex} \cite{pdf:biblatex}; e o quarto é um arquivo \texttt{.bib} que traz um conjunto de exemplos de referências para o \texttt{biblatex-abnt} \cite{bib:biblatex-abnt}. Esse último é bastante útil para consultar, porém, o resultado de saída na impressão sempre deve levar como base a NBR 6023:2018, então use os exemplos sempre com alguma certa cautela\footnote{A mesma recomendação é válida para os próprios exemplos dados neste modelo, pois, erros podem ter passado despercebidos.}.

\section{Referênciação de elementos do texto}\label{section:referencia}
Um aspecto básico, e muitas vezes necessário, é a referênciação de elementos presentes pelo seu texto, como figuras ou seções. O comando que permite criar essa referência é o comando \verb|\label|\footnote{\label{rodape:exemplo-ref}Uma dica é criar um padrão nos nomes dos rótulos, para facilitar a identificação e evitar o risco de repetir nomes. Por exemplo, iniciando com o nome do tipo de elemento que será referenciado e depois o nome sobre o elemento, utilizando dois pontos para separá-los, como \texttt{tabela:nome-tabela} ou \texttt{figura:nome-figura}.}, 
e, apesar de ser mais comum ver este comando presente em figuras ou tabelas, também serve para referenciar seções, notas de rodapé e outros elementos que possuam contadores, como equações, apêndices ou outros elementos criados pelo usuário\footnote{Elementos criados pelo usuário, como os quadros foram, também funcionarão se configurados corretamente.}. 
Depois, a citação é realizada através do comando \verb|\ref|, porém, é importante citar de que o comando do rótulo apenas seleciona o número do elemento, necessitando escrever separadamente o nome do elemento citado, como, por exemplo, citando a nota de rodapé \ref{rodape:exemplo-ref}, a Figura \ref{figura:marca-iffar}, o Anexo \ref{anexo:teste} ou a seção \ref{section:referencia}\footnote{Nesse aspecto, a padronização dos nomes, citados na nota \ref{rodape:exemplo-ref}, facilita bastante a identificação do elemento quando citado, para não se confundir na escrita.}. %ou a alínea \ref{alinea:exemplo-ref}. % O \ref para alíneas faz vir com o ) no nome, então não é tão útil

\section{Referências bibliográficas}\index{Referencias bibliograficas@Referências bibliográficas}
O gerenciamento das referências bibliográficas é realizado através do \verb|biblatex|, e o mesmo apresenta um conjunto de funções que podem ser utilizadas para facilitar a escrita. A primeira, e bastante importante, é que as obras da seção de referências só serão impressas se invocadas ao menos uma vez no documento através de citação. Outra funcionalidade que possui é a de gerenciar os anos das obras quando elas possuem mesmo ano e mesmos autores, que precisam ter letras adicionadas ao final. E, para realizar as citações, existe um grande conjunto de comandos para citar as obras.

\subsection{Tipos e exemplos de citações}
\subsubsection{Citação indireta}\index{Referencias bibliograficas@Referências bibliográficas!Citacao indireta@Citação indireta}
A citação indireta, que ocorre pela interpretação própria das palavras dos autores, pode ser realizada de duas formas: \verb|\cite|; ou, para citar o autor diretamente no texto, \verb|\textcite|:

\verb|\cite{livro:dom-casmurro}|

\cite{livro:dom-casmurro}\\

\verb|\textcite{livro:dom-casmurro}|

\textcite{livro:dom-casmurro}

\paragraph{Citação simultânea de múltiplas obras}
Existem situações em que também é necessário citar simultaneamente mais de uma obra e, nesses casos, também podem ser realizadas de duas formas: \verb|\cites|; ou, citando diretamente no texto, \verb|\textcites|:

\begin{verbatim}
    \cites{livro:abnt-nbr-14724:2011}
          {livro:abnt-nbr6023:2018}
          {livro:iffar-guia-normalizacao-2022}
\end{verbatim}%
\cites{livro:abnt-nbr-14724:2011}{livro:abnt-nbr6023:2018}
{livro:iffar-guia-normalizacao-2022}\\

\begin{verbatim}
    \textcites{livro:abnt-nbr-14724:2011}
              {livro:abnt-nbr6023:2018}
              {livro:iffar-guia-normalizacao-2022}
\end{verbatim}%
\textcites{livro:abnt-nbr-14724:2011}{livro:abnt-nbr6023:2018}
{livro:iffar-guia-normalizacao-2022}\\

Contudo, é importante citar que o \verb|biblatex| não realiza o ordenamento alfabético das múltiplas obras citadas, sendo função do autor do trabalho realizar esse controle na ordenação das entradas.

\subsubsection{Citação direta e citação direta longa}\index{Referencias bibliograficas@Referências bibliográficas!Citacao direta@Citação direta}\index{Referencias bibliograficas@Referências bibliográficas!Citacao direta@Citação direta!Citacao longa@Citação longa}
Para citações de mais de três linhas, é necessário adicioná-las a um novo parágrafo, com 4cm de recuo, espaçamento simples e fonte tamanho 10, sem a presença de aspas. No \LaTeX{}, isso é alcançado através do comando \verb|\blockcquote|, do pacote \verb|csquotes|, que ajusta automaticamente para o formato de citação longa quando ultrapassa as três linhas. Porém, aqui existe um alerta: o comando utilizado recomendado é o \texttt{blockcquote} e não o comando \texttt{blockquote}, tendo, então, a letra \textbf{c} entre \texttt{block} e \texttt{quote}.

Os dois comandos, \verb|\blockcquote| e \verb|\blockquote|%
\footnote{Sendo justo, não consegui fazer o \texttt{blockquote} funcionar corretamente com as citações do \texttt{biblatex}. Por isso recomendo o uso do \texttt{blockcquote}, que além de funcionar certinho, permite adicionar outras informações.}, 
apresentam a mesma funcionalidade, contudo, o \verb|\blockcquote| apresenta três argumentos para citações. Isso permite adicionar informações adicionais anteriores e posteriores à fonte, como o número das páginas exigido pela ABNT nas citações diretas. A grande vantagem de utilizar esse comando, é que ele cria o ambiente de citação de forma automática, ou seja, se o texto inserido possuir até três linhas, a citação aparece, já contendo as aspas, no corpo do parágrafo, mas, se apresentar mais de três linhas, é ajustado para um parágrafo único, com o recuo e as outras formatações necessárias. 
% O comando \verb|\blockquote| pode ser utilizado quando a autoria é citado antes no parágrafo, apesar de não ser impedido o uso do \verb|\blockcquote| sem os dados de citação.

Por conta da característica do comando de se ajustar automaticamente, a citação pode ser inserida no meio do parágrafo e, se ela tiver até três linhas, \blockcquote[p. 58]{livro:iffar-guia-normalizacao-2022}{é caracterizado como uma citação direta curta, não requerendo qualquer alteração na configuração do texto}\footnote{Lembre-se de que as aspas duplas utilizadas nas obras originais precisam ser trocadas por aspas simples quando estiverem no trecho citado.}\footnote{Outra opção é apenas adicionar o texto entre aspas, citando a obra pelo comando padrão após as aspas.}, porém, se possuir mais de três linhas, torna-se um parágrafo separado:
\blockcquote[tradução nossa, grifo nosso]{filme:bee-movie}{De acordo com todas as leis conhecidas da aviação, \textbf{não há como uma abelha poder voar}. Suas asas são muito pequenas para tirar seu pequeno corpo gordo do chão. A abelha, obviamente, voa de qualquer maneira, porque abelhas não se importam com o que humanos pensam ser impossível}

\subsubsection{Citação apud}\index{Referencias bibliograficas@Referências bibliográficas!Apud}
O \texttt{biblatex-abnt} introduz um conjunto de comandos para citações apud, contudo, os comandos adicionam as duas fontes na bibliografia. Isso contradiz o estabelecido no Guia de Normalização, que indica que a fonte original, ao qual não se tem acesso, não é adicionada. Porém, esse problema é resolvido ao adicionar o campo \verb|keywords = {apud}| no registro da fonte original\footnote{Observe o registro do campo \texttt{keywords} do \texttt{apud:obra-original} no arquivo \texttt{bibliografia.bib}.}, já que a impressão da bibliografia foi configurada para filtrar os registros dessa maneira. Por exemplo, citando tem-se o resultado esperado, com a fonte original \texttt{apud:obra-original} não aparecendo na lista de referência.

Os comandos presentes para as citações apud são \verb|\apud| e \verb|\textapud|:

\verb|\apud{apud:obra-original}{apud:obra-acessada}|

\apud{apud:obra-original}{apud:obra-acessada}\\

\verb|\textapud{apud:obra-original}{apud:obra-acessada}|

\textapud{apud:obra-original}{apud:obra-acessada}\\

Também existem alternativas para caso não se tenha ou não seja desejado ter o registro da fonte original no arquivo de bibliografia \texttt{.bib}:

\verb|\apud[IBGE, 1993][]{livro:abnt-nbr-14724:2011}|

\apud[IBGE, 1993][]{livro:abnt-nbr-14724:2011}\\

\verb|IBGE \cite[1993 apud][]{livro:abnt-nbr-14724:2011}|

IBGE \cite[1993 apud][]{livro:abnt-nbr-14724:2011}

\subsubsection{Citação de outros elementos da referência}
Quando busca-se citar alguns elementos diferentes da obra, como o seu título ou seu ano, podem ser utilizados comandos específicos para realizar tal função. Isso facilita o trabalho, por não necessitar realizar a escrita manualmente, diminuindo as chances de erros caso seja necessário atualizar algum dado da obra. Dos exemplos a seguir, muitos dos comandos imprimem o ano junto, contudo, para a correta citação, por exemplo, de um título de uma obra, ainda é necessário citar os autores e o ano.

Para citar o título da obra, utiliza-se \verb|\citetitle|, ou \verb|\citefield|, que também possui outras utilidades. Exemplo: 

\verb|\citetitle{livro:dom-casmurro}| ou \\\verb|\citefield{livro:dom-casmurro}{title}|

\citetitle{livro:dom-casmurro} ou \citefield{livro:dom-casmurro}{title}\\

Para citar apenas o ano, \verb|\citeyear|, por exemplo: 

\verb|\citeyear{livro:dom-casmurro} ou \citeyear*{livro:dom-casmurro}|

\citeyear{livro:dom-casmurro} ou \citeyear*{livro:dom-casmurro}\\

Para citar apenas os nomes dos autores existem várias opções, a primeira é o \verb|\citeauthor|, por exemplo: 

\verb|\citeauthor{livro:dom-casmurro} ou \citeauthor*{livro:dom-casmurro}|\\

Também pode-se utilizar o \verb|\citename|, com alguns parâmetros de configurações: 

\verb|\citename{livro:dom-casmurro}{author}| ou \\\verb|\citename{livro:dom-casmurro}[given-family]{author}|

\citename{livro:dom-casmurro}{author} ou \citename{livro:dom-casmurro}[given-family]{author}\\

Ou, então, utilizar o comando criado para este modelo, \verb|\citeauthorfull|\footnotemark:
\footnotetext{O funcionamento pleno deste comando em todas as situações não é garantido pois não foi amplamente testado. Para a sua criação, foi realizada a adaptação de dois códigos distintos e, sendo bem honesto, não entendo exatamente bem como funciona a parte lógica do \texttt{biblatex}.}

\verb|\citeauthorfull{livro:dom-casmurro}|

\citeauthorfull{livro:dom-casmurro}\\

E também existe a opção de citar determinados campos dos registros das referências, como já mostrado anteriormente, através do comando \verb|\citefield|\footnote{Nem todos os campos funcionam, pois alguns são listas que, logo, também possuem comando específico para impressão.}:

\verb|\citefield{lei:brasil-11.892-2008}{titleaddon}|

\citefield{lei:brasil-11.892-2008}{titleaddon}

\subsubsection{Elementos adicionais na citação}
Existem muitas situações em que é necessário adicionar outras informações sobre a citação, e isso é facilmente resolvido no \verb|biblatex|. Utilizando-se de colchetes, informações sobre a citação podem ser adicionadas, como o número da página, essencial para citações diretas, a expressão ``tradução nossa'' para citações diretas traduzidas ou, então, artigos, parágrafos e incisos de documentos jurídicos. Por exemplo:

\verb|\textcite[p. 239]{livro:dom-casmurro}|

\textcite[p. 239]{livro:dom-casmurro}\\

Ou:

\verb|\cite[Art. 1º, I ]{lei:brasil-11.892-2008}|

\cite[Art. 1º, I ]{lei:brasil-11.892-2008}\\

Para as citações simultâneas ou citações apud segue-se o mesmo padrão ao utilizar os colchetes: %
\begin{verbatim}
  \cites[<info. fonte 1>]{<fonte 1>}
  [<info. fonte 2>]{<fonte 2>}...
\end{verbatim}

\section{Exemplos de referências bibliográficas}
A seguir são apresentados exemplos de referências para variados tipos de documentos, porém, não leve os exemplos dados como os de melhor qualidade. Para alguns exemplos foi necessário realizar um conjunto de adaptações, que fogem da boa prática, para conseguir o resultado desejado. Será alertado no exemplo quando isso ocorrer, mas, recomenda-se sempre uma certa cautela ao seguir os exemplos, além de que alguns podem ter passado despercebido na revisão, para receber o alerta. Além de que, é claro, sempre é recomendado que se procure seguir as boas práticas contidas nas documentações dos pacotes sobre referências.

% ABNT 7.1
\subsection{Monografia no todo} % 7.1

% ABNT 7.1.1
\subsubsection{Livro e/ou folheto}% 7.1.1
  \exEssencial
  \singlecite{livro:luck-2010:essencial}{}

  \exComplementar
  \singlecite{livro:luck-2010:complementar}{}

  \exOutros
  \singlecite{livro:dom-casmurro}{}
  \singlecite{livro:abnt-nbr6023:2018}{}

% ABNT 7.1.2
\subsubsection{Trabalhos acadêmicos} % 7.1.2
  \exEssencial
  \singlecite{tese:rodrigues-2009:essencial}{}

  \exComplementar
  \singlecite{tese:rodrigues-2009:complementar}{}

  \exOutros
  \singlecite{tese:aguiar-2009}{}
  \singlecite{tcc:alves-2008}{}

% ABNT 7.2
\subsection{Monografia no todo em meio eletrônico} % 

% ABNT 7.2.1
\subsubsection{Documentos em meio eletrônico} % 7.2.1
  \exEssencial
  \singlecite{livro:koogan-1998}{\footnote{Adaptação em \texttt{editortype}.}}

  \exOutros
  \singlecite{livro:godinho-2014}{}
  
% ABNT 7.2.2
\subsubsection{Documentos disponíveis online} % 7.2.2
  \exEssencial
  \singlecite{livro:bavaresco-2011:essencial}{\footnote{Adaptação em \texttt{editortype}.}}

  \exComplementar
  \singlecite{livro:bavaresco-2011:complementar}{\footnote{Adaptação em \texttt{editortype} e \texttt{pages}.}}

  \exOutros
  \singlecite{livro:iffar-guia-normalizacao-2022}{\footnote{Adaptação em \texttt{editortype}.}}
  \singlecite{tese:coelho-2009}{}
  \singlecite{tese:santos-2018}{}

% ABNT 7.3
\subsection{Parte de monografia} % 7.3
  \exEssencial
  \singlecite{parte:romano:essencial}{\footnote{Adaptação em \texttt{editortype}.}}

  \exComplementar
  \singlecite{parte:romano:complementar}{\footnote{Adaptação em \texttt{editortype}.}}

  \exOutros
  \singlecite{parte:manfroi-2010}{\footnote{Adaptação em \texttt{editortype}. Atenção no sobrenome do coordenador terminado em Filho.}}
  \singlecite{parte:santos-1994}{}

% ABNT 7.4
\subsection{Parte de monografia em meio eletrônico} % 7.4
  \exEssencial
  \singlecite{parte:cancer-2010}{\footnote{Adaptação em \texttt{authortype} e \texttt{editortype}.}}

  \exOutros
  \singlecite{parte:politica-1998}{}

% ABNT 7.5
\subsection{Correspondência} % 7.5
  \exEssencial
  \singlecite{correspondencia:pilla:essencial}{\footnote{O mais correto seria utilizar o tipo \texttt{@letter}, mas, aparentemente, não foi implementado o tipo para as normas da ABNT.}}

  \exComplementar
  \singlecite{correspondencia:pilla:complementar}{}

% ABNT 7.6
\subsection{Correspondência disponível em meio eletrônico} % 7.6
  \exEssencial
  \singlecite{correspondencia:lispector}{}

% ABNT 7.7
\subsection{Publicação periódica} % 7.7

% ABNT 7.7.1
\subsubsection{Coleção de publicação periódica} % 7.7.1
  \exEssencial
  \singlecite{revista:ibge:essencial}{\footnote{Atenção no ano da obra utilizado.}}

  \exComplementar
  \singlecite{revista:ibge:complementar}{}

  \exOutros
  \singlecite{revista:nursing-1998}{\footnote{É muito comum o \& passar despercebido e causar erro se não for utilizado o seu comando.}}
  \singlecite{revista:nursing-1929}{}

% ABNT 7.7.2
\subsubsection{Coleção de publicação periódica em meio eletrônico} % 7.7.2
  \exEssencial
  \singlecite{revista:acta-1997:essencial}{\footnote{Adaptação em \texttt{issn}.}}

  \exComplementar
  \singlecite{revista:acta-1997:complementar}{}

  \exOutros
  \singlecite{revista:ufsc}{}

% ABNT 7.7.3
\subsubsection{Parte de coleção de publicação periódica} % 7.7.3
  \exEssencial
  \singlecite{revista:ibge:parte}{}

% ABNT 7.7.4
\subsubsection{Fascículo, suplemento e outros} % 7.7.4
  \exEssencial
  \singlecite{revista:tres}{}

  \exOutros
  \singlecite{revista:fgv}{\footnote{Adaptação grande de tipo e em \texttt{titleaddon} e \texttt{publisher}.}}

% ABNT 7.7.5
\subsubsection{Artigo, seção e/ou matéria de publicação periódica} % 7.7.5
  \exEssencial
  \singlecite{artigo:lucca-2009}{\footnote{O separador de períodos de datas foi modificado para ser uma barra, através do uso de um \texttt{options} criado para este modelo. Então agora é possível definir, usando a boa prática, períodos como jul./ago. e 2022/2023, ou qualquer outra data. Olhe este registro no arquivo \texttt{bib} para visualizar as alternativas para isso.}}

  \exOutros
  \singlecite{artigo:rocke-1985}{\footnote{Adaptação em \texttt{issue}. Novamente, o correto seria utilizar os campos de divisão do ano, porém também traduzindo os campos, além de utilizar a barra.}}
  \singlecite{artigo:costa-1998}{\footnote{Atenção para o uso de subtítulo no nome do periódico.}}

% ABNT 7.7.6
\subsubsection{Artigo, seção e/ou matéria de publicação periódica em meio eletrônico} % 7.7.6
  \exEssencial
  \singlecite{artigo:vieira-1994}{\footnote{Atenção no uso de \texttt{yeardivision} para imprimir a estação do ano.}}

  \exOutros
  \singlecite{artigo:alexandrescu-2009}{\footnote{Adaptação em \texttt{location}, para a impressão do sine loco em um artigo.}}
  \singlecite{artigo:moore-2018}{\footnote{Adaptação em \texttt{location}.}}
  \singlecite{artigo:antunes-2018}{\footnote{Adaptação em \texttt{location}.}}

% ABNT 7.7.7
\subsubsection{Artigo e/ou matéria de jornal} % 7.7.7
  Atenção no posicionamento da página do artigo/matéria, pois, \blockcquote[p. 15]{livro:abnt-nbr6023:2018}{quando não houver seção, caderno ou parte, a paginação do artigo ou matéria precede a data.}

  \exEssencial
  \singlecite{noticia:otta-2010}{\footnote{Adaptação em \texttt{note}.}}
  
  \exOutros
  \singlecite{noticia:credito-2014}{\footnote{Adaptação em \texttt{pages}.}}

% ABNT 7.7.8
\subsubsection{Artigo e/ou matéria de jornal em meio eletrônico} % 7.7.8
  \exEssencial
  \singlecite{noticia:verissimo-2010}{}

  \exOutros
  \singlecite{noticia:professores-2010}{\footnote{Adaptação em \texttt{note}.}}

% ABNT 7.8
\subsection{Evento} % 7.8
A referenciação de eventos é simples, mas possui algumas limitações seguindo alguns exemplos dados na NBR 6023:2018. Se for apenas um evento simples, de Anais e coisas do tipo, não tem complicação, porém, se for como alguns exemplos dados na NBR 6023, que apresentam dois eventos em uma mesma referência\footnote{Interpretação do autor.}, surgem problemas. Nessa situação não foi possível identificar uma forma de fazer o registro seguindo as boas práticas.

% ABNT 7.8.1
\subsubsection{Evento no todo em monografia} % 7.8.1
Observação importante para registro de eventos no todo, que usam o tipo \texttt{@proceedings}: é necessário definir o \texttt{sorttitle} com o mesmo texto de \texttt{eventtitle} ou, então, utilizar \texttt{organization} para colocar o nome do evento. Isso acontece por causa de uma limitação na ordenação dos registros, que não permite definir uma ordenação específica para um tipo de entrada, como é o caso de \texttt{@proceedings}.

  \exEssencial
  \singlecite{evento:chemical-1984}{\footnote{Adaptação em \texttt{titleaddon}, para não deixar em negrito os colchetes com reticências. Isso talvez leve a problemas caso a intenção seja de imprimir seu título no texto.}}

  \exOutros
  \singlecite{evento:ines-2009}{\footnote{Adaptação em \texttt{number} e \texttt{eventyear}. Atenção no uso do \texttt{addendum} ao invés de \texttt{note}}}

% ABNT 7.8.2
\subsubsection{Evento no todo em publicação periódica} % 7.8.2
  \exEssencial
  \singlecite{evento:pet-2006}{\footnote{Adaptação em \texttt{number}, \texttt{eventyear}, \texttt{venue} e \texttt{publisher}. Exemplo extremamente imprudente, não siga-o.}}

  \exOutros
  \singlecite{evento:olericultura-2001}{\footnote{Adaptação em \texttt{number}, \texttt{eventyear}, \texttt{venue} e \texttt{publisher}. Exemplo extremamente imprudente, não siga-o.}}

% ABNT 7.8.3
\subsubsection{Evento no todo em meio eletrônico} % 7.8.3
  \exEssencial
  \singlecite{evento:ufpe-1996}{}

  \exOutros
  \singlecite{evento:hotel-2004}{\footnote{Atenção no uso de colchetes em torno do ano de publicação.}}
  \singlecite{evento:soja-2009}{\footnote{Adaptação em \texttt{number} e \texttt{eventyear}. Outro exemplo que vai totalmente contra às boas práticas.}}

% ABNT 7.8.4
\subsubsection{Parte de evento} % 7.8.4
% ABNT 7.8.4.1
\paragraph{Parte de evento em monografia} % 7.8.4.1
  \exEssencial
  \singlecite{parte-evento:brayner-1994}{}

  \exOutros
  \singlecite{parte-evento:martin-neto-1997}{\footnote{Adaptação em \texttt{note}.}}

% ABNT 7.8.4.2
\paragraph{Parte de evento em publicação periódica} % 7.8.4.2
  \exEssencial
  \singlecite{parte-evento:goncalves-2006}{}

% ABNT 7.8.5
\paragraph{Parte de evento em meio eletrônico} % 7.8.5
  \exEssencial
  \singlecite{parte-evento:guncho-1998}{}

  \exOutros
  \singlecite{parte-evento:badke-2006}{}
  \singlecite{parte-evento:goncalves-2000}{\footnote{Adaptação em \texttt{note}.}}

% ABNT 7.9
\subsection{Patente} % 7.9
O caso de referenciar patentes é um tanto confuso: a ABNT define como referenciar, porém, não foi identificado como que uma patente deve ser citada no texto, por causa da questão do ano, que possui data de depósito e data de concessão. Considerando essa incerteza, sobre a citação do ano\footnote{O \texttt{biblate-abnt} também adiciona o campo de data logo depois de \texttt{location}, dificultando identificar a boa prática.}, serão disponibilizadas duas alternativas de adaptação utilizando o campo de data, além de uma outra sem data definida\footnote{O \texttt{biblatex-abnt} imprime a data antes do número da patente, dificultando a identificação de como conciliar a boa prática e a correta impressão.}.
  \exEssencial
  \singlecite{patente:bertazzoli}{\footnote{Alternativa 1: utilizando a data de depósito para referência. Adaptação em \texttt{titleaddon}.}}

  \exOutros
  \singlecite{patente:vicente}{\footnote{Alternativa 2: utilizando a data de concessão para referência. Adaptação em \texttt{titleaddon}.}}
  \singlecite{patente:oliveira}{\footnote{Alternativa 3: não utilizando data. Atenção no uso do campo \texttt{number}.}}

% ABNT 7.10
\subsection{Patente em meio eletrônico} % 7.10
  \exEssencial
  \singlecite{patente:galembeck}{\footnote{Ao não utilizar o campo de data, o \LaTeX{} utiliza a data de acesso na referência.}}

% ABNT 7.11
\subsection{Documento jurídico} % 7.11

% ABNT 7.11.1
\subsubsection{Legislação} % 7.11.1
  \exEssencial
  \singlecite{livro:constituicao-rs}{}

  \exOutros
  \singlecite{livro:constituicao-brasil}{}
  \singlecite{lei:brasil-10.406-2002}{}

% ABNT 7.11.2
\subsubsection{Legislação em meio eletrônico} % 7.11.2
  \exEssencial
  \singlecite{livro:constituicao-brasil-url-1}{\footnote{Atenção no ano utilizado na citação. Pela minha interpretação, parece ser o ano em que ocorreu a última emenda constitucional, porém, é necessário verificar isto.}}

  \exOutros
  \singlecite{lei:curitiba}{\footnote{Atenção no fato da formatação de leis ser diferente entre meio físico e meio eletrônico, não é utilizado o tipo \texttt{@legislation}.}}
  \singlecite{inbook:brasil-vade-mecum}{}
  \singlecite{lei:brasil-11.892-2008}{}

% ABNT 7.11.3
\subsubsection{Jurisprudência} % 7.11.3
  \exEssencial
  \singlecite{jurisprudencia:stf-2006}{}

  \exOutros
  \singlecite{jurisprudencia:stj-2007}{}

% ABNT 7.11.4
\subsubsection{Jurisprudência em meio eletrônico} % 7.11.4
  \exEssencial
  \singlecite{jurisprudencia:stf-eletronico}{\footnote{Adaptação em \texttt{publisher} e uso de \texttt{options}.}}

  \exOutros
  \singlecite{jurisprudencia:stj-eletronico}{}
  \singlecite{jurisprudencia:stj-2010-eletronico}{\footnote{Adaptação em \texttt{publisher} e uso de \texttt{options}.}}

% ABNT 7.11.5
\subsubsection{Atos administrativos normativos} % 7.11.5
  \exEssencial
  \singlecite{ato:rio-aviso-2005}{}

  \exOutros
  \singlecite{ato:banco-central-2007}{}
  \singlecite{ato:fundeb}{}

% ABNT 7.11.6
\subsubsection{Atos administrativos normativos em meio eletrônico} % 7.11.6
  \exEssencial
  \singlecite{ato:fazenda-url}{}

  \exOutros
  \singlecite{ato:ufu-2007}{}

% ABNT 7.12
\subsection{Documentos civis e de cartórios} % 7.12
  \exEssencial
  \singlecite{documento:certidao-essencial}{\footnote{Adaptação em \texttt{location} e uso de \texttt{nosn} em \texttt{options} (o mesmo poderia ser feito utilizando \texttt{nosl} e adaptando pelo \texttt{publisher}).}}

  \exComplementar
  \singlecite{documento:certidao-complementar}{}

% ABNT 7.13
\subsection{Documento audiovisual} % 7.13

% ABNT 7.13.1
\subsubsection{Filmes, vídeos, entre outros} % 7.13.1
  \exEssencial
  \singlecite{filme:toxico-1983}{\footnote{Adaptação em \texttt{titleaddon}. O último sobrenome das pessoas é formatado em maiúsculas, então é necessário adaptar os nomes das pessoas nos filmes e outras obras que compartilhem essa característica.}}

  \exOutros
  \singlecite{filme:central-1998}{}

% ABNT 7.13.2
\subsubsection{Filmes, vídeos, entre outros em meio eletrônico} % 7.13.2
  \exEssencial
  \singlecite{filme:john-2003}{}

  \exOutros
  \singlecite{filme:blade-runner}{}
  \singlecite{filme:book-2010}{}
  \singlecite{filme:bee-movie}{}
  \singlecite{serie:grande-familia}{}
  \singlecite{serie:fullmetal}{}

% ABNT 7.13.3
\subsubsection{Documentos sonoros} % 7.13.3
  \exEssencial
  \singlecite{sonoro:mosaico-2005}{}

  \exOutros
  \singlecite{sonoro:beethoven}{}
  \singlecite{sonoro:cid-moreira-2010}{}
  \singlecite{sonoro:gomes-2011}{}

% ABNT 7.13.4
\subsubsection{Parte de documento sonoro} % 7.13.4
  \exEssencial
  \singlecite{sonoro:jura-1977}{}

  \exOutros
  \singlecite{sonoro:toque-1988}{}

% ABNT 7.13.5
\subsubsection{Documento sonoro em meio eletrônico} % 7.13.5
  \exEssencial
  \singlecite{sonoro:podcast-2010}{}

  \exOutros
  \singlecite{sonoro:anticast-2013}{}

% ABNT 7.14
\subsection{Partitura} % 7.14

%ABNT 7.14.1
\subsubsection{Partitura impressa} % 7.14.1
  \exEssencial
  \singlecite{partitura:xenakis-1980}{}

  \exOutros
  \singlecite{partitura:brahms-1977}{}

%ABNT 7.14.2
\subsubsection{Partitura em meio eletrônico} % 7.14.2
  \exEssencial
  \singlecite{partitura:beethoven-1863}{}

  \exOutros
  \singlecite{partitura:gonzaga-1997}{}

% ABNT 7.15
\subsection{Documento iconográfico} % 7.15
  \exEssencial
  \singlecite{imagem:kobayashi-1980}{\footnote{Atenção para o uso do \texttt{noslsn} no campo \texttt{option}.}}

  \exOutros
  \singlecite{imagem:levi-1997}{}
  \singlecite{imagem:ferrari-1990}{}

% ABNT 7.16
\subsection{Documento iconográfico em meio eletrônico} % 7.16
  \exEssencial
  \singlecite{imagem:houte-2010}{\footnote{Adaptação em \texttt{titleaddon}.}}

  \exOutros
  \singlecite{imagem:floripa-2011}{\footnote{Adaptação em \texttt{nameaddon}.}}

% ABNT 7.17
\subsection{Documento cartográfico} % 7.17
  \exEssencial
  \singlecite{cartografico:igc-1994}{}

  \exOutros
  \singlecite{cartografico:cesp-1981}{}

% ABNT 7.18
\subsection{Documento cartográfico em meio eletrônico} % 7.18
  \exEssencial
  \singlecite{cartografico:museum-2000}{\footnote{Adaptação em \texttt{note} devido a um bug, já que o ponto de interrogação é um caractere de pontuação, que leva a um tratamento diferente, o que confunde o \texttt{biblatex} na hora de formatar a referência.}}

  \exOutros
  \singlecite{cartografico:inpe-2014}{}

% ABNT 7.19
\subsection{Documento tridimensional} % 7.19
  \exEssencial
  \singlecite{escultura:duchamp-1918-essencial}{}

  \exComplementar
  \singlecite{escultura:duchamp-1918-complementar}{}

  \exOutros
  \singlecite{escultura:bule-1800}{\footnote{Atenção em \texttt{year}, onde os hífens devem estar separados para o \LaTeX{} não transformá-los em um travessão.}}

% ABNT 7.20
\subsection{Documento de acesso exclusivo em meio eletrônico} % 7.20
  \exEssencial
  \singlecite{software:apple-2017}{}

  \exOutros
  \singlecite{software:ufp-word-1998}{}
  \singlecite{software:got-2017}{\footnote{Adaptação em \texttt{titleaddon}.}}
  \singlecite{tweet:oliveira-2020}{}
  \singlecite{site:wikipedia-2010}{\footnote{Por algumas limitações do \texttt{biblatex}, os colchetes adicionados se tornam um problema ao imprimir errado o ano quando citado.}\footnote{A NBR 6023:2018 \cite{livro:abnt-nbr6023:2018} dá este exemplo para um artigo da Wikipédia, porém, o Guia de Normalização \cite[p. 75]{livro:iffar-guia-normalizacao-2022} exemplifica de uma maneira diferente, mais simples.}}
  \singlecite{facebook:fbn-2015}{}
  \singlecite{blog:cid-2011}{\footnote{Atenção no uso de \texttt{options}.}}
  \singlecite{mensagem:almeida-2002}{}
  % \singlecite{tweet:guizin-2020}{}
  % \singlecite{site:latex-impressora}{\footnote{Essa é a mesma formatação utilizada para citar artigos da Wikipédia.}}
