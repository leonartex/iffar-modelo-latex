\chapter{Configurações e discussões sobre o modelo}\label{capitulo:configuracao}
\section{Preenchimento de dados}
Uma das primeiras coisas que deve ser feita neste modelo é o preenchimento dos dados utilizados pelo modelo. Isso é realizado no arquivo \texttt{elementos/DADOS.tex} e, este arquivo, foi pensado como o ponto central para configuras algumas páginas pré-textuais, como capa e folha de rosto, pensando em facilitar a alteração desses registros. Se a pessoa quiser, pode realizar o preenchimento dos dados manualmente em cada um desses arquivos.

A pessoa deve iniciar indicando se o seu trabalho é do tipo TCC ou não, indicando no comando \verb|\isTCC| o valor \textbf{1 = TCC} ou \textbf{0 = Tese ou Dissertação}. Esse comando é necessário para o modelo alterar a capa e folha de rosto, além de alternar entre folha de aprovação, para TCCs, ou folha de certificação, para teses e dissertações. Após isso, podem ser preenchidos alguns dados gerais utilizados entre as páginas, seguido pela escolha do preenchimento de dados: para TCC; ou Tese e Dissertação. Por último, são preenchidos os dados sobre a sua banca e coordenação do seu curso/Programa.

Importante ressaltar de que, no caso de não existir coorientação, o comando \verb|\coorientador| seja deixado totalmente vazio. Pois o modelo verifica o preenchimento desse valor para adicionar ou remover suas citações nas páginas. Além disso, realize a alteração no tamanho do espaço vertical dado para os campos de assinatura\footnote{Apesar de você adicionar o determinado tamanho para os campos, ele não vai ter especificamente aquele tamanho utilizado. Isso acontece por causa que ainda são somados aos espaços entre parágrafos, além do espaço de uma linha do filete de assinatura.}. Isso é importante, pois, o espaço foi pensado no modelo utilizando a fonte Arial, de tamanho maior se comparado com a Times New Roman, e considerando a existência de coorientação no trabalho. E, por último, não tenha medo de realizar ajustes manualmente em algumas dessas páginas pré-textuais, caso a lógica utilizada com os comandos para o preenchimento automático não funcione tão bem.

\section{Ajustando outros espaçamentos}
Outros elementos que podem receber o ajuste em determinados espaçamentos são: sumário; lista de siglas; lista de símbolos; e a margem das notas de rodapé. Neles foram utilizados a unidade \textbf{ex}, que usa como referência o tamanho de uma letra \textit{x}, mas você pode utilizar qualquer outra medida aceita pelo \LaTeX. Isso é algo extremamente opcional de se realizar, porém, podem existir situações que necessitam disso, pois o espaço utilizado ou é muito grande, ou muito pequeno.

Para ajustar o espaçamento do \textbf{sumário}: Acesse o arquivo \\\texttt{config/sumario.sty} e altere o comando \verb|\espacoAlinhamentoSumario|. O \texttt{10ex} representa o espaço utilizado, que foi definido considerando um possível número de seções, que chegam até o quinto nível.

Para ajustar as listas de \textbf{símbolos} e de \textbf{siglas}: Acesso os respectivos arquivos de cada uma dessas duas listas, na pasta \texttt{elementos/}, alterando o espaço definido nos comandos \verb|\simboloLargura| e \verb|\siglaLargura|, respectivamente.

O ajuste do espaçamento do algarismo da nota de rodapé pode ser alterado no arquivo \texttt{config/rodape.sty}. Só o comando \verb|\footnotemargin| no arquivo e escolher o tamanho do espaçamento para alinhamento. Não foi identificado no Guia de Normalização e nem em normas da ABNT, ou então passou despercebido, a indicação de qual deveria ser a margem entre o algarismo e a nota, então, a configuração utilizou o número que melhor se enquadrou.

\section{Escolha de fonte}
No início do arquivo \texttt{main.tex} você pode optar entre utilizar fonte Arial ou Times New Roman. Isso é realizado pelo o comando \verb|\setmainfont|, você apenas precisa localizar esses comandos, comentar uma dessas linhas e descomentar a fonte escolhida no início do arquivo. Uma condição opcional é, no caso do uso da Arial, a utilização da Andale Mono como a opção de fonte monoespaçada, utilizada em algumas situações. Caso opte por utilizá-la e esteja utilizando o \LaTeX{} localmente em sua máquina, baixe e instale a fonte\footnote{Local por onde se pode baixá-la: \url{https://www.cufonfonts.com/font/andale-mono}}. Para o Overleaf nenhum trabalho adicional é necessário, pois, ele já apresenta suporte à fonte. Se não for utilizar a fonte monoespaçada, apenas comente a sua linha abaixo da definição da Arial.

\subsection{Configuração de fontes para matemática}
Para o caso de utilização de equações e fórmulas, ainda é necessária a seleção da fonte utilizada nesses elementos. No modelo, é utilizado o pacote \texttt{unicode-math} para a configuração de fontes e, dependendo da seleção anterior de fonte (Arial ou Times New Roman), também é necessário selecionar as suas respectivas fontes. Se você não for utilizar os elementos matemáticos do \LaTeX, pode ignorar essa seção e remover a importação\footnote{Recomendo que apenas comente as linhas} dos pacotes \texttt{amsmath} e \texttt{unicode-math}, além de suas definições de fontes matemáticas abaixo\footnote{O pacote \texttt{amssymb}, e seu arquivo de configuração abaixo, também pode ser removido, mas existe o seu uso na definição de alguns caracteres, para permitir a utilização do caractere diretamente no texto dos arquivos.}.

Para o caso de Times New Roman, é utilizada a Latin Modern Math como fonte matemática, com a substituição da Times New Roman para os números e letras latinas e gregas. Já, para o caso da utilização da Arial, ocorre o mesmo processo de substituição, mas utilizando a fonte matemática sem serifa Fira Math. Se estiver utilizando o \LaTeX{} localmente, assim como no caso da Andale Mono, será necessário baixar a fonte e instalá-la em seu computador\footnote{https://github.com/firamath/firamath/releases}.

Ao compilar, o \LaTeX irá enviar um conjunto de alertas, por usar a fonte Times New Roman ou Arial em alguns conjuntos como \textit{mathfont}. Não foi identificado problemas que ocorram por causa do uso dessas fontes, porém, se você identificar problemas em suas equações, você pode: no caso de uso da Times New Roman, comentar as linhas de \verb|\setmathfont| que utilizem a Times, pois a Latin Modern já é uma fonte semelhante o suficiente; e no caso da Arial, também comentar as linhas de \verb|\setmathfont| utilizando Arial, porém, a Fira Math pode não ser tão semelhante a ela, então, sugere-se a troca da fonte matemática\footnote{As fontes suportadas pelo Overleaf estão em \textcite{site:overleaf-fontes-2023}. E no link a seguir estão um conjunto de exemplos de fontes com suporte matemático: \textcite{site:tex-fontes-2022}}.

\section{Criação de novos elementos de ilustração e listas}\label{section:novos-elementos}
Uma característica que pode ser necessária em seu trabalho é a criação de novos elementos de ilustração para listagem. Por exemplo, você pode necessitar criar um elemento \iindex{Gráfico}{Ilustracao@Ilustração!Grafico@}, ou Fotografias, junto com suas respectivas listas. E isso é facilmente realizavel no arquivo \texttt{main.tex}: abaixo da importação do pacote \texttt{newfloat} estão contidas as instruções para realizar tal ajuste.

\section{Utilização do \LaTeX{} no PC ou pelo navegador}
Para utilizar o \LaTeX{} você basicamente tem duas opções: instalar na sua máquina; ou utilizar pelo navegador. Utilizando pelo navegador, o editor mais conhecido é o Overleaf\footnote{\url{https://www.overleaf.com/}}\footnote{Basicamente um Google Docs do \LaTeX.}, que, para este modelo, você precisa garantir que esteja sendo usado o compilador XeLaTeX, com o documento aberto, clicando em Menu, no canto superior esquerdo, e selecionando o XeLaTeX como o compilador. E, para utilizar localmente, na sua máquina, existe a opção de utilizar o próprio VS Code como o editor de texto. Essa última opção é bastante vantajosa, por permitir você realizar o controle do seu diretório \textit{git}, se quiser criar, não depender de uma conexão estável de internet\footnote{Eu nunca tive internet decente e, na moral, se você tiver internet horrorosa também, utilize pelo VS Code. Além de que é muito gostosinho de escrever com o auxílio de sugestão de código.} e ter o benefício da usabilidade da interface da IDE e da própria extensão\footnote{Sendo honesto, eu consigo ser muito mais produtivo pelo VS Code do que usando o Overleaf pelo navegador, porém, entendo as vantagens do Overleaf, de permitir compartilhar com outras pessoas e editar simultaneamente. Apesar que, em teoria, a edição simultânea pode ser realizada por causa que existem extensões no VS Code para realizar programação de forma remota e simultânea (VS Code humilha o Overleaf. Venha, junte-se a nós).}.

Para utilizar o \LaTeX{} pelo VS Code, baixe e instale uma instância de \LaTeX, como a MiKTeX\footnote{É a que eu utilizo e recomendo. Ela já vem com alguns pacotes úteis instalados por padrão.} e instale a extensão LaTeX Workshop no VS Code. Após isso, ainda é necessário realizar a configuração da extensão: clique \texttt{ctrl+shift+p}, selecione ``\textit{Preferences: Open User Settings (JSON)}'' e copie as configurações que estão no arquivo \texttt{docs/settings.json}, porém, copie com cuidado, selecionando com atenção apenas as configurações relacionadas à extensão e ao \LaTeX. A própria extensão LaTeX Workshop possui uma \textit{wiki} que pode servir de auxílio para os ajustes.

\section{Oneside e Twoside}
Uma das configurações possíveis de se realizar no documento é a criação de um documento que utilize os dois versos de uma folha, como um livro. Apesar do \LaTeX{} ter a limitação de não indicar à impressora que ela deve imprimir nos dois lados da folha \cite{site:latex-impressora}, o modelo foi configurado para poder ser utilizado dessa forma, mesmo que não seja algo comum e seja, aparentemente, bastante opcional\footnote{Não garanto que funcione 100\%, não me cobrem se der algo errado.}. E essa configuração ocorre ao optar, no \texttt{documentclass} do documento, entre \texttt{oneside} e \texttt{twoside}.

O Guia de Normalização recomenda, para o caso de utilizar o verso das folhas, que os elementos pré-textuais e pós-textuais sempre iniciem no anverso da folha (a parte da frente). Aparentemente, nesse caso, realmente é apenas uma recomendação, segue quem quer e só se utilizar a opção \texttt{twoside}. Para seguir essa recomendação, foi configurado o uso do \verb|\cleardoublepage|\footnote{O modelo detecta qual configuração está sendo usada para ativar o comando} no fim de cada elemento pré-textual\footnote{Existe uma exceção envolvendo a folha de rosto, que deve ter a ficha catalográfica ou de identificação de obra no seu verso}, e textual, no qual o usuário deverá adicionar o comando em cada um de seus capítulos. Porém, existe um \textit{bug} identificado, que está ocorrendo no último capítulo, que o comando está criando uma página em branco a mais. Nesse caso, apenas deixe comentado o comando, pois as referências, ao se utilizar como capítulo, cria-se sempre em uma nova página\footnote{E se ocorrer algum outro tipo de bug assim, só comente o comando que cria essa página em branco, pois, ele não é tão essencial. Terminar um capítulo \texttt{clearpage} ou \texttt{cleardoublepage} é mais uma boa prática do que necessidade, já que o capítulo sempre inicia em uma nova página.}.


%%%%%%%%%%%%%%%%%%%%%%%%%%%%%%%%%%
% Isso aqui é só um elemento estético, para separar o meu alerta. Por favor, não utilize no seu trabalho algo assim, a ABNT iria chorar se o fizesse.
\vspace{\baselineskip}
\begin{Center}
    \rule{12cm}{1pt}
\end{Center}

\textbf{ATENÇÃO:} É muito comum que um documento no \LaTeX, durante a compilação, dê algum erro e ele, na verdade, seja originado por causa de algum caractere diferente do habitual. Alguns caracteres especiais, como o \_, que deve ser utilizado através do comando \verb|\textunderscore ou \_|\footnote{Na Tabela \ref{tabela:exemplo-caracteres} mostra-se um conjunto de caracteres especiais, alguns deles podem levar a erros se digitados diretamente no texto.}, levam a erros na compilação, e podem estar contidos em alguma de suas referências utilizadas, e geralmente estão. Por isso recomenda-se bastante atenção no registro das referências bibliográficas, com exceção de URLs, pois, muitas vezes, esses erros ocorrem de forma silenciosa, sem dar um alerta do que pode estar causando ele (principalmente se você estiver utilizando o método de compilação \textbf{XeLaTeX $\times$ Biber $\times$ XeLaTeX $\times$ XeLaTeX}, que basicamente oculta mensagens de erro quando eles estão acontecendo na bibliografia, por causa das compilações seguintes).