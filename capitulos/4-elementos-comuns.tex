\chapter{Exemplos de elementos comuns} % Exemplo de seção primária
  Neste capítulo é apresentado um conjunto de elementos comuns que provavelmente serão utilizados em algum momento por todos os autores de trabalhos acadêmicos, como figuras e tabelas. Por questão de praticidade, no texto, não será discutido o código que leva à impressão de determinado elemento, talvez com algumas pequenas exceções. O objetivo deste capítulo é apresentar seus exemplos, realizando algumas observações sobre cada elemento, porém, tendo no código do capítulo, no arquivo \verb|capitulos/4-elementos-comuns.tex|, a exemplificação de como utilizar determinado elemento no \LaTeX{}, com comentários, para maiores explicações.

\section{Exemplos de seções} % Exemplo de seção secundária
  Essa aqui é um exemplo de seção secundária, mas as seções, ao todo, podem apresentar até 5 (cinco) níveis de subdivisão. A seção primária, como pode ter sido observado, são os capítulos, mas, a seguir, apresentam-se as outras seções.

  \subsection{Exemplo de seção terciária}
  \lipsum[1]

  \subsubsection{Exemplo de seção quaternária}
  \lipsum[2]

  \paragraph{Exemplo de seção quinária} % Utiliza-se paragraph para o nível hierarquico inferior à subsubsection. Contudo, foi necessário customizá-lo para apresentar corretamente como um nível de seção
  \lipsum[3]

  \paragraph{Exemplo de título com indicação numérica que, ao ocupar mais de uma linha, deve ser, a partir da segunda linha, alinhado abaixo da primeira letra da primeira palavra do título}
  \lipsum[4]

\section{Notas de rodapé}
  As notas de rodapé devem apresentar tamanho 10, alinhamento à esquerda e espaçamento justificado, além de, ao utilizar mais de uma linha, deve ser alinhado abaixo da primeira letra da primeira palavra da primeira linha\footnote{\lipsum*[5]}. No \LaTeX{} existe um conjunto de comandos para utilizá-las, por exemplo, a nota pode ser adicionada no meio do parágrafo\footnote{Exatamente como a primeira nota}, porém, também pode ter apenas a sua marcação realizada no parágrafo\footnotemark, permitindo preencher o seu texto posteriormente.
    \footnotetext{Exemplo de nota usando a marcação.}
  
  É importante relatar que existe um problema ao utilizar apenas a marcação: o comando para adicionar o texto à nota utiliza apenas o número da nota de rodapé mais recente\footnotemark, então, se você adicionar duas \verb|\footnotemark| seguidas\footnotemark, deixando os dois \verb|\footnotetext| para depois, será utilizada apenas o número da última nota criada, com as duas notas apresentando o mesmo número. 
    \footnotetext{Nota relativa à primeira marcação.}
    \footnotetext{Nota relativa à segunda marcação.}
  Dessa forma, a recomendação é utilizar o \verb|\footnotemark| apenas quando o parágrafo possuir apenas uma nota ou adicionar o texto da marcação após o término da frase e não do parágrafo. Ou, então, não utilizar, realizando a quebra de linha se quiser deixar mais formatadinho o código no \LaTeX\footnote{Um novo parágrafo só é criado quando existirem duas quebras de linhas seguidas, ou seja, uma linha em branco.}.

\section{Alíneas e subalíneas}
  As alíneas\index{Alineas@Alíneas} sempre devem vir a partir de um parágrafo que termine com dois pontos antes de iniciar a lista:
  \begin{alinea}
    \item a matéria da alínea começa por letra minúscula, exceto quando se tratar de substantivos próprios, e termina em ponto e vírgula, com exceção da última, que termina em ponto final;
    \item o trecho final da seção correspondente, anterior às alíneas, termina em
    dois pontos;
    \item as alíneas são ordenadas por letras minúsculas seguidas de parênteses utilizando-se letras dobradas quando esgotadas as 26 (vinte e seis) letras que compõem o alfabeto brasileiro;\label{alinea:exemplo-ref}
    \item as letras indicativas das alíneas são recuadas em relação à margem
    esquerda, alinhadas com o parágrafo;
    \item o texto da alínea deve terminar em dois pontos, se houver subalíneas\index{Alineas@Alíneas!Subalineas@Subalíneas}: 
    \begin{subalinea}
      \item a matéria da subalínea começa por letra minúscula e termina em ponto e
      vírgula, e a última subalínea deve terminar em ponto final, se não houver
      alínea subsequente; 
      \item são iniciadas por travessão seguido de espaço;
      \item devem apresentar recuo em relação à alínea;
      \item a segunda linha e as demais que se seguem no texto da subalínea
      começam sob a primeira letra do texto da própria subalínea.
    \end{subalinea}
    \item a segunda linha e as demais que se seguem no texto da alínea começam
    sob a primeira letra do texto da própria alínea.
  \end{alinea}
E se não for quebrada a linha, para iniciar um novo parágrafo, o \LaTeX{} continuará no mesmo parágrafo, não realizando a indentação, contudo, não foi identificado pelo autor se essa característica é algo desejado. Não foi encontrada nenhuma norma na ABNT que tratasse especificamente sobre isso, nem algum exemplo no Guia de Normalização. Na dúvida, é só quebrar a linha duas vezes, para começar um novo parágrafo

  % \subsection{Outros exemplos de listas numeradas}
    % \begin{leiParagrafo}
    %   \item a matéria da alínea começa por letra minúscula, exceto quando se tratar de     substantivos próprios, e termina em ponto e vírgula, com exceção da última, que termina em ponto final; 
    %   \item o trecho final da seção correspondente, anterior às alíneas, termina em
    %   dois pontos;
    %   \item asdasdsada
    %   \item sdsdsdaa
    % \end{leiParagrafo}

\section{Equações e fórmulas}
  Existe um grande número de pacotes para a impressão de elementos e símbolos matemáticos\footnote{Vou ser bem honesto, isso aqui não é nem um pouco a minha área. Estou citando aqui por causa que é um ponto importante, mas não tenho ideia direito do que estou falando.}. Este modelo, por exemplo, já tem importado o pacote \texttt{amssymb}, para a adição de símbolos, e o \texttt{amsmath}\footnote{Adicionei só para ajudar quem não conhece sobre, se você não utilizar nada ``matemático demais'', esse pacote pode ser tranquilamente removido.}, para maiores funcionalidade com elementos matemáticos, mas nada impede a adição de outros pacotes. Porém, deve-se atentar com o  conflito de pacotes. Dependendo do que os pacotes fazem, principalmente se tiverem funcionalidades muito semelhantes, pode ocorrer conflito, levando a erros na compilação. E isto é uma recomendação geral sobre pacotes. Não é necessário ter receio de adicionar outros pacotes, porém, eles devem ser adicionados com atenção, visualizando quais pacotes o modelo já utiliza\footnote{Muitos pacotes apresentam em suas documentações uma lista de pacotes conhecidos por darem conflitos. Alguns conflitos são de total incompatibilidade, porém, outros podem ser apenas em determinados elementos, que vão se comportar de uma maneira errônea.}.

  Para a apresentação de expressões matemáticas, o \LaTeX{} apresenta muitas maneiras para se utilizar, podendo ocorrer de dois modos: \textit{inline} e \textit{display}. O primeiro é para expressões que fazem parte de um parágrafo e o segundo é utilizado em um bloco de texto exclusivo para a expressão. Seguem-se exemplos:

  No modo \textit{inline} pode ser escrito expressões de várias formas\footnote{No código são utilizadas várias maneiras de se apresentar expressões matemáticas \textit{inline}.}, como: \(E=mc^2\); $ 2 \times 2 = 4 $; e, também, \begin{math} x^2 + y^2 = z^2 \end{math}. Contudo, o Guia de Normalização estabelece que equações e fórmulas devem aparecer \blockcquote[p. 23]{livro:iffar-guia-normalizacao-2022}{destacadas no texto, a fim de facilitar sua leitura}, para isto existe o modo \textit{display}\footnote{Eu não pesquisei como deixar as expressões mais alinhadas à esquerda, deixando de ficar centralizada, dando um pequeno recuo para o destaque. Uma alternativa é utilizar o modo \textit{inline} isolado como um único parágrafo, assim a aplica a indentação e realiza o destaque.}. E, para o modo \textit{display}, também existem múltiplas formas de adicionar uma expressão ao texto\footnote{Observe no código as diversas formas de se criar uma expressão matemática em modo \textit{display}.}:
  \[ x^n + y^n = z^n \]
  
  Ou:
  \begin{displaymath}
    \sqrt{x^2+1}
  \end{displaymath}
  
  Ou, então:
  \begin{equation}
    e^{\pi i} + 1 = 0
  \end{equation}

  Também existe o ambiente \texttt{equation*}, provido pelo pacote \texttt{amsmath}\footnote{Esse exemplo foi extraído do pacote, para mostrar as possibilidades. Recomendo a leitura da documentação do pacote, caso tenha interesse.}:
  \begin{equation*}
    \left.\begin{aligned}
    B’&=-\partial\times E,\\
    E’&=\partial\times B - 4\pi j,
    \end{aligned}
    \right\}
    \qquad \text{Maxwell’s equations}
  \end{equation*}

  O \texttt{amsmath} também provê um grande número de funcionalidades, para realizar o alinhamento das expressões\footnote{Recomendação de leitura: \url{https://www.overleaf.com/learn/latex/Aligning_equations_with_amsmath}}:
  
  \begin{equation}
    \begin{split}
      A& = \frac{\pi r^2}{2} \\
       & = \frac{1}{2} \pi r^2
    \end{split}
  \end{equation}
  
  Ou:
  \begin{multline}
    p(x) = 3x^6 + 14x^5y + 590x^4y^2 + 19x^3y^3\\ 
    - 12x^2y^4 - 12xy^5 + 2y^6 - a^3b^3
  \end{multline}

  A ABNT prevê também, e de maneira opcional, a possibilidade de realizar a numeração das equações. Como é observável, o próprio elemento \texttt{equation} do \LaTeX{} já apresenta sua própria numeração, além de alguns outros elementos do \texttt{amsmath} porém, também existem maneiras de desativar essa função\footnote{Posso ter entendido errado, mas o equation* do \texttt{amsmath} tem a exata diferença de não ter numeração.}.

\section{Figuras, tabelas e quadros}
  Sobre elementos ilustrativos, é importante destacar que o comando utilizado para criar esses elementos é apenas um invólucro, onde você pode colocar o conteúdo que quiser dentro deles. E esse invólucro é apenas responsável por dar o ``tratamento'' como determinado item, e aí, dentro dele, são realizados os comandos que realmente adicionam os itens pertinentes ao respectivo elemento, como uma imagem, no caso de figuras.
  
\subsection{Figuras}
  Para as figuras, podem ser utilizados os formatos de imagens mais populares, como \verb|.png| ou \verb|.jpg|\footnote{É válido pesquisar quais são todos os formatos aceitos.}, porém, além de imagens, também podem ser adicionados documentos \verb|.pdf|. A vantagem de utilizar um \textit{Portable Document Format} (PDF) é que os textos contidos não perderão a qualidade, por não passarem por compressão, ao imprimir as palavras diretamente como texto, como ilustrado na Figura \ref{figura:pdf}. Porém, apesar de aceitar formatos populares de imagens, nativamente (e pelo pacote utilizado), imagens \textit{Scalable Vector Graphics} (SVG) não são possiveis de serem utilizadas. Inclusive, até existem pacotes que adicionam a capacidade de imprimir SVGs nos documentos, porém, eles possuem um conjunto de limitações\footnotemark. A seguir, na Figura \ref{figura:marca-iffar}, é apresentado um exemplo com a marca do IFFar.
      \footnotetext{Dependendo do programa que for utilizado, existe a opção de exportar a figura como PDF, logo, imprimindo corretamente a imagem.}

  \begin{figure}[H]
    \Centering\singlespacing
    \caption{Marca do IFFar}
    \label{figura:marca-iffar} % Esse é o comando que permite referenciar a figura (ou qualquer outro elemento que você criar)
    % Podem ser utilizados vários argumentos para adicionar a figura em seu devido tamanho, como: scale; width; ou height. No caso de scale, a escala, adiciona o valor em decimal representando a porcentagem da escala, já width ou height utiliza-se uma unidade de medida, como cm, px ou pt.
    \includegraphics[width=10cm]{marca-iffar.png}

    \footnotesize
    Fonte: \textcite{site:iffar-identidade-visual-2021}.
  \end{figure}

  \begin{figure}[H]
    \Centering\singlespacing
    \caption{Exemplo utilizando documento PDF}
    \label{figura:pdf}
    \includegraphics[scale=1.0]{exemplo.pdf}

    \footnotesize
    Fonte: elaborado pelo autor (2023).
  \end{figure}
  
\subsection{Tabelas}
  Como citado anteriormente, os elementos das figuras, tabelas e quadros são apenas um invólucro, então, para tabelas (e quadros), podem ser utilizados um conjunto de alternativas para gerarem-se as tabelas. No caso deste modelo, pode ser utilizadas para gerar as tabelas as opções de \verb|tabular|, nativo do LaTeX, e \verb|tabularx|,adicionado através do pacote homônimo, que permite criar tabelas com tamanhos variáveis de colunas. Existem também outros pacotes que permitam gerar tabelas e, então, pode ser útil pesquisar sobre eles. Na Tabela \ref{tabela:barbetta-2017-p248} é apresentada uma tabela utilizando \verb|tabular| e, na Tabela \ref{tabela:exemplo-caracteres}, utilizando \verb|tabularx|.

\begin{table}[H]
  \Centering\singlespacing

  \caption{Classificação de pessoas segundo o nível de instrução e colaboração
  com a coleta seletiva do lixo
  }
  \label{tabela:barbetta-2017-p248}
  \begin{tabular}{l|c|c} % As pipelines "|" indicam onde deve haver uma linha separando as células e as letras indicam o alinhamento do conteúdo das células da coluna
    \hline % Adiciona um traço horizontal na tabela
    \multirow{2}{*}{Nível de instrução} & \multicolumn{2}{c}{Colabora com a coleta seletiva do lixo} \\ % Por causa do \multicolumn, essa linha não apresenta o último "&" para a última célula e o "\\" finaliza a linha
    \cline{2-3} % Já que o \hline nessa linha iria desconfigurar por causa do \multirow, é necessário criar uma linha que inicie de determinada coluna e termine em outra coluna determinada
    & \parbox{3cm}{\Centering sim} & \parbox{3cm}{\Centering não} \\  % Por conta do \multirow na linha acima, a primeira célula dessa linha fica em branco
    \hline
    nenhum ou fundamental & 22 & 13 \\
    médio & 33 & 34 \\
    superior & 39 & 36\\
    \hline
   \end{tabular}

\hspace{\fill}

\footnotesize

Fonte: \textcite[p. 248]{livro:barbetta-2017}.
\end{table}

\begin{table}[H]
  \Centering\singlespacing

  \caption{Exemplos de caracteres especiais no LaTeX}
  \label{tabela:exemplo-caracteres}

  \begin{tabularx}{0.85\textwidth} { 
    >{\centering\arraybackslash}X 
    | >{\centering\arraybackslash}X}
    \hline
    Comando    & Caractere   \\ \hline
    \verb|\&|  & \&          \\ \hline
    \verb|\$|  & \$          \\ \hline
    \verb|\{|  & \{          \\ \hline
    \verb|\}|  & \}          \\ \hline
    \verb|\_|  & \_          \\ \hline
    \verb|$\backslash$|  & $\backslash$          \\ \hline
    \verb|\#|  & \#          \\ \hline
    \verb|\textemdash|       & \textemdash        \\ \hline
    \verb|\textendash|       & \textendash        \\ \hline
    %\verb|\textbar|          & \textbar           \\ \hline
    n\verb|\textordmasculine{}|          & n\textordmasculine{}           \\ \hline
    \verb|\textordfeminine{}|          & \textordfeminine{}           \\ \hline
    \verb|\textregistered|   & \textregistered    \\ \hline
    \verb|\copyright|        & \copyright         \\
    \hline
  \end{tabularx}

  \hspace{\fill}

  % A quebra de linha é essencial para o espaço horizontal ser corramente ocupado, não ocorrendo do texto de fonte ficar posicionado na posição errada, não centralizada
  \footnotesize
  Fonte: elaborado pelo autor (2023).

  Nota: há a exceção dos caracteres º e ª, pois eles foram definidos como símbolos \textit{unicode} aceitos, no arquivo \texttt{config/unicode-char.sty}, contudo, talvez nem seja necessário realizar tal configuração, pois, o XeLaTeX compila em UTF-8. O mesmo pode ser realizado para alguns outros exemplos, se optado, como o \textregistered{} ou \copyright. Além disso, o --- e o -- também podem ser realizados através de 3 e 2 híphens seguidos, respectivamente.
\end{table}

\subsection{Quadros}
  Para apresentar um quadro, utiliza-se a mesma lógica das tabelas. Na verdade, as únicas coisas que diferenciam um quadro de uma tabela é o termo utilizado para criar o elemento, \verb|\begin{quadro}|, e o fato que quadros são todo cobertos por bordas, enquanto que tabelas não apresentam bordas nas extremidades laterais. E essa borda é impressa ao adicionar os \textit{pipelines} (``|'') nas extremidades, ao definir as colunas do \verb|tabular[x]| (ex.: \verb|\begin{tabular}{|\textbf{|}\verb\c|c|c\\textbf{|}\verb|}|). A seguir, no Quadro \ref{quadro:exemplo}, isso pode ser visualizado.

  \begin{quadro}[H]
    \Centering\singlespacing  
    \caption{Exemplo de quadro com múltiplas linhas e colunas}
    \label{quadro:exemplo}
    \begin{tabularx}{0.65\textwidth}
      {|c|r|l|X|} % Definição das colunas
      \hline
      Coluna 1  & Coluna 2  & Coluna 3  & Coluna 4 \\
      \hline
      \multicolumn{2}{|c|}{Itens 1 e 2} & Item 3    & Item 4 \\
      \hline
      Item 5    & Item 6    & Item 7   & \multirow{2}{*}{Itens 8 e 12} \\
      \cline{1-3}
      \multirow{3}{*}{\begin{sideways}9,13,17\end{sideways}}& Item 10   & Item 11  &                                   \\
      \cline{2-4}
      & Item 14   & \multicolumn{2}{c|}{\multirow{2}{*}{Itens 15, 16, 19 e 20}} \\
      \cline{2-2}
      & Item 18  & \multicolumn{2}{c|}{}\\
      \hline
    \end{tabularx}
  
    \hspace{\fill}
  
    % A quebra de linha é essencial entre o \hspace e outras linhas
    \footnotesize
    Fonte: elaborado pelo autor (2023).
  \end{quadro}
