\chapter{Exemplos de elementos comuns}
\section{Notas de rodapé}
  \lipsum[10]\footnote{\lipsum*[11]}

\section{Alíneas e subalíneas}
  \lipsum[12]
  \begin{alinea}
    \item a matéria da alínea começa por letra minúscula, exceto quando se tratar de     substantivos próprios, e termina em ponto e vírgula, com exceção da última, que termina em ponto final; 
    \item o trecho final da seção correspondente, anterior às alíneas, termina em
    dois pontos;
    \item asdasdsada: 
    \begin{subalinea}
      \item a matéria da alínea começa por letra minúscula, exceto quando se tratar de     substantivos próprios, e termina em ponto e vírgula, com exceção da última, que termina em ponto final; 
      \item o trecho final da seção correspondente, anterior às alíneas, termina em
      dois pontos;
    \end{subalinea}
    \item sdsdsdaa
  \end{alinea}
  
  \lipsum[13]

  % \subsection{Outros exemplos de listas numeradas}
    % \begin{leiParagrafo}
    %   \item a matéria da alínea começa por letra minúscula, exceto quando se tratar de     substantivos próprios, e termina em ponto e vírgula, com exceção da última, que termina em ponto final; 
    %   \item o trecho final da seção correspondente, anterior às alíneas, termina em
    %   dois pontos;
    %   \item asdasdsada
    %   \item sdsdsdaa
    % \end{leiParagrafo}

\section{Equações e fórmulas}

\section{Figuras, tabelas e quadros}
  --- Falar que figure, table e quadro são apenas invólucros, que qualquer coisa pode ser adicionada dentro, que apenas indica que o elemento possuirá seu respectivo tratamento diferenciado na hora de listar e apresentar como tal
  
\subsection{Figuras}
  --- Falar os formatos de imagem suportados
  \begin{figure}[H]
    \Centering\singlespacing
    \caption{As dimensões da comunicação organizacional}
    \label{figura:marca-iffar}
    % Podem ser utilizados vários argumentos para adicionar a figura em seu devido tamanho, como: scale; width; ou height. No caso de scale, a escala, adiciona o valor em decimal representando a porcentagem da escala, já width ou height utiliza-se uma unidade de medida, como cm, px ou pt.
    \includegraphics[scale=0.35]{marca-iffar.png}\\
    \footnotesize
    Fonte: \textcite{iffar-identidade-visual-2021}.
  \end{figure}
  
\subsection{Tabelas}
\begin{table}[H]
  \Centering\singlespacing

  \caption{Posições do IFFar no \textit{ranking} do painel de dados da LAI}
  \label{tabela:cgu-iffar}
  \begin{tabularx}{12cm}
    {X r r} % Definição das colunas
    \hline
    \multicolumn{1}{c}{\textbf{Categoria}}    &   
    \multicolumn{1}{c}{\textbf{Posição}}  & 
    \multicolumn{1}{c}{\textbf{Informação adicional}} \\
    \hline

    Transparência ativa &
    252\textordmasculine{} &
    24,49\% em cumprimento \\
    %\hline

    Pedidos recebidos &
    192\textordmasculine{} &
    807 pedidos \\
    %\hline

    Tempo médio de resposta para pedidos de informação       &   
    154\textordmasculine{} &
    14,91 dias \\
    \hline

  \end{tabularx}

\hspace{\fill}

\footnotesize

Fonte: Gaba guba.
\end{table}

\subsection{Quadros}
--- Falar que é uma tabela com bordinha nos lados
\begin{quadro}[H]
  \Centering\singlespacing

  \caption{Posições do IFFar no \textit{ranking} do painel de dados da LAI}
  \label{quadro:cgu-iffar}
  \begin{tabularx}{12cm}
    {|X|r|r|} % Definição das colunas
    \hline
    \multicolumn{1}{|c|}{\textbf{Categoria}}    &   
    \multicolumn{1}{c}{\textbf{Posição}}  & 
    \multicolumn{1}{|c|}{\textbf{Informação adicional}} \\
    \hline

    Transparência ativa &
    252\textordmasculine{} &
    24,49\% em cumprimento \\
    %\hline

    Pedidos recebidos &
    192\textordmasculine{} &
    807 pedidos \\
    %\hline

    Tempo médio de resposta para pedidos de informação       &   
    154\textordmasculine{} &
    14,91 dias \\
    \hline

  \end{tabularx}

\hspace{\fill}

\footnotesize

Fonte: Gaba guba.
\end{quadro}
