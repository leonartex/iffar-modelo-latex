\chapter{Introdução}
Este é um modelo para trabalhos acadêmicos do Instituto Federal de Educação, Ciência e Tecnologia Farroupilha (IFFar) construído no \LaTeX{}. Por trabalho acadêmico, refere-se aos Trabalhos de Conclusão de Curso (TCC), teses e dissertações. E, sobre o modelo, por ser construído no \LaTeX{}, ele é fácilmente adaptável em algumas aspectos, permitindo estender suas implementações. Outra característica deste modelo, também, é que é necessário realizar o preenchimento de alguns dados, para que o modelo faça a configuração de determinados elementos pré-textuais, porém, isso será detalhado no Capítulo \ref{capitulo:configuracao}.

Uma observação necessária: esse modelo não substitui o Guia de Normalização do IFFar. A implementação do modelo, obviamente, levou como base os aspectos estabelecidos pelo Guia e normas técnicas da Associação Brasileira de Normas Técnicas (ABNT), contudo, o modelo não é insuscetível a falhas. Então utilize-o com consciência disso.

Para a apresentação do restante deste modelo, são abordados: no segundo capítulo, algumas configurações necessárias e observações sobre o uso do modelo e do \LaTeX{}; no terceiro, explicações sobre os elementos de trabalhos acadêmicos presentes no modelo; no quarto, a apresentação de alguns elementos comuns, presentes nos textos; e no quinto, e último, capítulo, a explicação sobre uso de citações e referências no \LaTeX{}.