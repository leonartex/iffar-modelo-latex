\chapter{Introdução}
Este é um modelo para trabalhos acadêmicos do Instituto Federal de Educação, Ciência e Tecnologia Farroupilha (IFFar) construído no \LaTeX{}. Por trabalho acadêmico, refere-se aos Trabalhos de Conclusão de Curso (TCC), teses e dissertações. E, sobre o modelo, por ser construído no \LaTeX, ele é fácilmente adaptável em algumas aspectos, permitindo estender suas implementações. Um outro aspecto desse modelo, também, é que, de início, é necessário realizar o preenchimento de alguns dados, para a configuração de alguns elementos pré-textuais, porém, isso será detalhado no Capítulo \ref{capitulo:configuracao}.

A parte textual dos trabalhos acadêmicos, o Guia de Normalização do IFFar estabelece a obrigatoriedade da apresentação de três elementos:
    \begin{alinea}
        \item introdução;
        \item desenvolvimento;
        \item e conclusão.
    \end{alinea}
Essa é a definição lógica do que o conteúdo deve abranger e a ordem que deve ser apresentada, porém, não é a definição dos capítulos que a pessoa deve apresentar. Isso significa que a pessoa deve seguir essa estrutura lógica no seu conteúdo, mas tem total liberdade para definir os seus capítulos. 

Para facilitar para quem está com dúvida, e aqui é uma recomendação apenas do autor deste modelo, a estrutura de capítulos seguida pode ser, por exemplo, e apenas se quiser\footnote{Cada área pode ter suas particularidades e também recomenda-se conversar com que lhe orienta para discutir como deve ser estruturado o seu trabalho}: 
    \begin{alinea}
        \item Introdução;
        \item Fundamentação teórica (ou Revisão da literatura);
        \item Metodologia;
        \item Análise e discussão dos resultados;
        \item Considerações finais (ou Conclusões).
    \end{alinea}

Para a apresentação do restante deste modelo, são abordados: no segundo capítulo, algumas configurações necessárias no modelo; no terceiro, explicações sobre os elementos opcionais e obrigatórios para os trabalhos acadêmicos; no quarto, a apresentação de elementos comuns presentes nos textos; no quinto, a apresentação sobre uso de citações e referências; e, por último, as referências.